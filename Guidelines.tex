%%
% Copyright (c) 2017 - 2023, Pascal Wagler;
% Copyright (c) 2014 - 2023, John MacFarlane
%
% All rights reserved.
%
% Redistribution and use in source and binary forms, with or without
% modification, are permitted provided that the following conditions
% are met:
%
% - Redistributions of source code must retain the above copyright
% notice, this list of conditions and the following disclaimer.
%
% - Redistributions in binary form must reproduce the above copyright
% notice, this list of conditions and the following disclaimer in the
% documentation and/or other materials provided with the distribution.
%
% - Neither the name of John MacFarlane nor the names of other
% contributors may be used to endorse or promote products derived
% from this software without specific prior written permission.
%
% THIS SOFTWARE IS PROVIDED BY THE COPYRIGHT HOLDERS AND CONTRIBUTORS
% "AS IS" AND ANY EXPRESS OR IMPLIED WARRANTIES, INCLUDING, BUT NOT
% LIMITED TO, THE IMPLIED WARRANTIES OF MERCHANTABILITY AND FITNESS
% FOR A PARTICULAR PURPOSE ARE DISCLAIMED. IN NO EVENT SHALL THE
% COPYRIGHT OWNER OR CONTRIBUTORS BE LIABLE FOR ANY DIRECT, INDIRECT,
% INCIDENTAL, SPECIAL, EXEMPLARY, OR CONSEQUENTIAL DAMAGES (INCLUDING,
% BUT NOT LIMITED TO, PROCUREMENT OF SUBSTITUTE GOODS OR SERVICES;
% LOSS OF USE, DATA, OR PROFITS; OR BUSINESS INTERRUPTION) HOWEVER
% CAUSED AND ON ANY THEORY OF LIABILITY, WHETHER IN CONTRACT, STRICT
% LIABILITY, OR TORT (INCLUDING NEGLIGENCE OR OTHERWISE) ARISING IN
% ANY WAY OUT OF THE USE OF THIS SOFTWARE, EVEN IF ADVISED OF THE
% POSSIBILITY OF SUCH DAMAGE.
%%

%%
% This is the Eisvogel pandoc LaTeX template.
%
% For usage information and examples visit the official GitHub page:
% https://github.com/Wandmalfarbe/pandoc-latex-template
%%

% Options for packages loaded elsewhere
\PassOptionsToPackage{unicode}{hyperref}
\PassOptionsToPackage{hyphens}{url}
\PassOptionsToPackage{dvipsnames,svgnames,x11names,table}{xcolor}
%
\documentclass[
  paper=a4,
  ,captions=tableheading
]{scrartcl}
\usepackage{amsmath,amssymb}
% Use setspace anyway because we change the default line spacing.
% The spacing is changed early to affect the titlepage and the TOC.
\usepackage{setspace}
\setstretch{1.2}
\usepackage{iftex}
\ifPDFTeX
  \usepackage[T1]{fontenc}
  \usepackage[utf8]{inputenc}
  \usepackage{textcomp} % provide euro and other symbols
\else % if luatex or xetex
  \usepackage{unicode-math} % this also loads fontspec
  \defaultfontfeatures{Scale=MatchLowercase}
  \defaultfontfeatures[\rmfamily]{Ligatures=TeX,Scale=1}
\fi
\usepackage{lmodern}
\ifPDFTeX\else
  % xetex/luatex font selection
\fi
% Use upquote if available, for straight quotes in verbatim environments
\IfFileExists{upquote.sty}{\usepackage{upquote}}{}
\IfFileExists{microtype.sty}{% use microtype if available
  \usepackage[]{microtype}
  \UseMicrotypeSet[protrusion]{basicmath} % disable protrusion for tt fonts
}{}
\makeatletter
\@ifundefined{KOMAClassName}{% if non-KOMA class
  \IfFileExists{parskip.sty}{%
    \usepackage{parskip}
  }{% else
    \setlength{\parindent}{0pt}
    \setlength{\parskip}{6pt plus 2pt minus 1pt}}
}{% if KOMA class
  \KOMAoptions{parskip=half}}
\makeatother
\usepackage{xcolor}
\definecolor{default-linkcolor}{HTML}{A50000}
\definecolor{default-filecolor}{HTML}{A50000}
\definecolor{default-citecolor}{HTML}{4077C0}
\definecolor{default-urlcolor}{HTML}{4077C0}
\usepackage[margin=2.5cm,includehead=true,includefoot=true,centering,]{geometry}
% add backlinks to footnote references, cf. https://tex.stackexchange.com/questions/302266/make-footnote-clickable-both-ways
\usepackage{footnotebackref}
\setlength{\emergencystretch}{3em} % prevent overfull lines
\providecommand{\tightlist}{%
  \setlength{\itemsep}{0pt}\setlength{\parskip}{0pt}}
\setcounter{secnumdepth}{5}
\ifLuaTeX
\usepackage[bidi=basic]{babel}
\else
\usepackage[bidi=default]{babel}
\fi
\babelprovide[main,import]{english}
% get rid of language-specific shorthands (see #6817):
\let\LanguageShortHands\languageshorthands
\def\languageshorthands#1{}
\usepackage{framed}
\usepackage{xcolor}
\let\oldquote=\quote
\let\endoldquote=\endquote
\colorlet{shadecolor}{gray!5}
\renewenvironment{quote}{\begin{shaded*}\begin{em}}{\end{em}\end{shaded*}}
\ifLuaTeX
  \usepackage{selnolig}  % disable illegal ligatures
\fi





\IfFileExists{bookmark.sty}{\usepackage{bookmark}}{\usepackage{hyperref}}
\IfFileExists{xurl.sty}{\usepackage{xurl}}{} % add URL line breaks if available
\urlstyle{same}



\hypersetup{
  pdftitle={NFDI4Microbiota Foundation Guidelines},
  pdflang={en},
  pdfkeywords={NFDI, guidelines, policy, open science, open
data, FAIR, Germany, open access},
  hidelinks,
  breaklinks=true,
  pdfcreator={LaTeX via pandoc with the Eisvogel template}}

\title{NFDI4Microbiota Foundation Guidelines}

\author{}
\date{version 0.2}




%%
%% added
%%


%
% for the background color of the title page
%
\usepackage{pagecolor}
\usepackage{afterpage}
\usepackage[margin=2.5cm,includehead=true,includefoot=true,centering]{geometry}

%
% break urls
%
\PassOptionsToPackage{hyphens}{url}

%
% When using babel or polyglossia with biblatex, loading csquotes is recommended
% to ensure that quoted texts are typeset according to the rules of your main language.
%
\usepackage{csquotes}

%
% captions
%
\definecolor{caption-color}{HTML}{777777}
\usepackage[font={stretch=1.2}, textfont={color=caption-color}, position=top, skip=4mm, labelfont=bf, singlelinecheck=false, justification=raggedright]{caption}
\setcapindent{0em}

%
% blockquote
%
\definecolor{blockquote-border}{RGB}{221,221,221}
\definecolor{blockquote-text}{RGB}{119,119,119}
\usepackage{mdframed}
\newmdenv[rightline=false,bottomline=false,topline=false,linewidth=3pt,linecolor=blockquote-border,skipabove=\parskip]{customblockquote}
\renewenvironment{quote}{\begin{customblockquote}\list{}{\rightmargin=0em\leftmargin=0em}%
\item\relax\color{blockquote-text}\ignorespaces}{\unskip\unskip\endlist\end{customblockquote}}

%
% Source Sans Pro as the default font family
% Source Code Pro for monospace text
%
% 'default' option sets the default
% font family to Source Sans Pro, not \sfdefault.
%
\ifnum 0\ifxetex 1\fi\ifluatex 1\fi=0 % if pdftex
    \usepackage[default]{sourcesanspro}
  \usepackage{sourcecodepro}
  \else % if not pdftex
    \usepackage[default]{sourcesanspro}
  \usepackage{sourcecodepro}

  % XeLaTeX specific adjustments for straight quotes: https://tex.stackexchange.com/a/354887
  % This issue is already fixed (see https://github.com/silkeh/latex-sourcecodepro/pull/5) but the
  % fix is still unreleased.
  % TODO: Remove this workaround when the new version of sourcecodepro is released on CTAN.
  \ifxetex
    \makeatletter
    \defaultfontfeatures[\ttfamily]
      { Numbers   = \sourcecodepro@figurestyle,
        Scale     = \SourceCodePro@scale,
        Extension = .otf }
    \setmonofont
      [ UprightFont    = *-\sourcecodepro@regstyle,
        ItalicFont     = *-\sourcecodepro@regstyle It,
        BoldFont       = *-\sourcecodepro@boldstyle,
        BoldItalicFont = *-\sourcecodepro@boldstyle It ]
      {SourceCodePro}
    \makeatother
  \fi
  \fi

%
% heading color
%
\definecolor{heading-color}{RGB}{40,40,40}
\addtokomafont{section}{\color{heading-color}}
% When using the classes report, scrreprt, book,
% scrbook or memoir, uncomment the following line.
%\addtokomafont{chapter}{\color{heading-color}}

%
% variables for title, author and date
%
\usepackage{titling}
\title{NFDI4Microbiota Foundation Guidelines}
\author{}
\date{version 0.2}

%
% tables
%

%
% remove paragraph indention
%
\setlength{\parindent}{0pt}
\setlength{\parskip}{6pt plus 2pt minus 1pt}
\setlength{\emergencystretch}{3em}  % prevent overfull lines

%
%
% Listings
%
%


%
% header and footer
%
      \usepackage[headsepline,footsepline]{scrlayer-scrpage}

    \newpairofpagestyles{eisvogel-header-footer}{
      \clearpairofpagestyles
      \ihead*{NFDI4Microbiota Foundation Guidelines}
      \chead*{}
      \ohead*{version 0.2}
      \ifoot*{}
      \cfoot*{}
      \ofoot*{\thepage}
      \addtokomafont{pageheadfoot}{\upshape}
    }
    \pagestyle{eisvogel-header-footer}

    
      
%%
%% end added
%%

\begin{document}

%%
%% begin titlepage
%%
      \begin{titlepage}
          \newgeometry{left=6cm}
    
    \newcommand{\colorRule}[3][black]{\textcolor[HTML]{#1}{\rule{#2}{#3}}}
    \begin{flushleft}
    \noindent
    \\[-1em]
    \color[HTML]{5F5F5F}
    \makebox[0pt][l]{\colorRule[FFB12D]{1.3\textwidth}{3pt}}
    \par
    \noindent

        {
      \setstretch{1.4}
      \vfill

              \noindent
        \includegraphics{nfdi4microbiota\_Logo\_new.png}
      
      \noindent {\huge \textbf{\textsf{NFDI4Microbiota Foundation
Guidelines}}}
            \vskip 2em
      \noindent {\Large \textsf{}}
      \vfill
      \textsf{version 0.2}
    }
    
    \end{flushleft}
    \end{titlepage}
    \restoregeometry
    \pagenumbering{arabic} 
  %%
%% end titlepage
%%

% \maketitle


{
\setcounter{tocdepth}{3}
\tableofcontents
\newpage
}
\hypertarget{motivation}{%
\section*{Motivation}\label{motivation}}
\addcontentsline{toc}{section}{Motivation}

NFDI4Microbiota acts as a central infrastructure hub for the national
microbiology community by providing services for research data
management, processing and publishing for scientists in the field. It
relies in turn on already established infrastructure and services
provided especially by de.NBI and NFDI4Microbiota associated partners
for hosting and expertise in data processing, quality assessment and
analysis.

These Guidelines shall act as basis for specific policies of all
NFDI4Mircobiota projects and services. NFDI partners and other consortia
are encouraged to use and adapt these guidelines according to their
specific needs under the CC-BY license for their own purpose.
NFDI4Microbiota associated partners aim to offer data handling and
storage, bioinformatic tools and services as well as training within the
scope of a centralized infrastructure to a broad scientific audience. In
order to provide a consistent experience, permanent availability and
long-term support following the
\href{https://www.go-fair.org/fair-principles/}{FAIR principles} and the
Open science concepts, offered services and members of NFDI4Microbiota
are supposed to adhere to the following guidelines, which are monitored
regularly by TA leads and in Use Case reviews internally.

\begin{quote}
The following items are meant to be short and concise descriptions of
community goals within NFDI4Microbiota that are communicated to partners
and users (on the website). Expanded explanations in italics are meant
as clarifications to NFDI4Microbiota members in help of fulfilling
milestones and implementing services (not necessarily shown to end users
or collapsed by default, to keep the guideline as brief as possible).
Further details can additionally be found e.g.~in the
\href{https://nfdi4microbiota.github.io/nfdi4microbiota-knowledge-base/}{NFDI4Microbiota
Knowledge Base}.
\end{quote}

\hypertarget{availability}{%
\section{Availability}\label{availability}}

\hypertarget{providing-software-and-documentation}{%
\subsection{Providing software and
documentation}\label{providing-software-and-documentation}}

Applications offered in the context of the NFDI4Microbiota are provided
free of charge for download in case of offline tools. Online tools are
provided for any researcher with reasonable quota of usage also free of
charge. Links to the applications with a short description and
accompanying documentation are available on
\url{www.nfdi4microbiota.de}.

\begin{quote}
We aim to use and encourage the usage of free of charge tools to enable
access for the broadest possible audience and anyone with a scientific
interest. Pay-walls and license fees or unnecessary registrations may
hinder the adoption and hence the re-usability considerably. Similarly,
registration with a dedicated account may deter potential users. Instead
services are encouraged to use existing Authentication and Authorization
Infrastructure (AAI) such as
\href{https://lifescience-ri.eu/ls-login.html}{LifeScience-RI}. Software
that is designed to work on a local machine should be designed to work
also without a network connection, where feasible and online-only tools
should offer reasonable computation and storage quota for users to
conduct meaningful analysis (not just example data).
\end{quote}

\hypertarget{open-source}{%
\subsection{Open source}\label{open-source}}

Software in this context is developed under an
\href{https://opensource.org/licenses/}{OSI-approved license} and the
source code is published in openly accessible code repositories, along
with appropriate documentation under an open license.

\begin{quote}
From experience, tools developed in academia sometimes lack in the
necessary documentation, are no longer maintained shortly after
publication or not available at all after personnel left or institute's
website links changed. In these cases, potentially useful tools become
useless at best or even a frustrating waste of time for following
generations of researchers. Openly hosted source code under a permissive
license enables others to troubleshoot, adapt or even take over the
development of a tool even in these cases. We recommend including the
relatively concise \href{https://opensource.org/license/mit/}{MIT
license}. Publishing the actual source of a tool on a public code
repository is necessary to prevent black-box results and preserve
resources persistently. We recommend code hosting on \url{github.com},
\url{gitlab.org}, {[}gitlab.eudat.eu{]}, {[}codeberg.org{]} or similar
services for software projects. NFDI4Microbiota specific projects are
hosted under the
\href{github.com/NFDI4Microbiota}{github.com/NFDI4Microbiota team
account}. Author should ideally also make use of the code deposition
feature of \href{zenodo.org}{Zenodo} and make sure that the git
repository is archived by \href{softwareheritage.org}{Software
Heritage}. We also encourage to apply good research software development
practices (including automated testing, semantic versioning,
machine-readable software management plans, allocation of DOI to
version, usage of Citation File Format (CFF) etc.).
\end{quote}

\hypertarget{hosting}{%
\subsection{Hosting}\label{hosting}}

Online provided services are supposed to be accessible with minimal
downtime. To assure the permanent availability, hosting containerized
services in the de.NBI cloud infrastructure and re-using established
high-availability services within the scientific community is
encouraged.

\begin{quote}
Even though we recognize that hosting web-tools on cloud services
requires additional effort, we highly encourage this step instead of or
additionally to hosting on local institute's machines to guarantee
scalability for high demand of compute and storage resources and provide
high availability. If done correctly, containerizing applications can
prevent a lot of future work and headaches. Further
\href{https://www.denbi.de/images/Service/deNBI_ELIXIR-DE_Gudielines_Services_version_20200729.pdf}{de.NBI
specific guidelines} for hosted services apply.
\end{quote}

\hypertarget{open-access-publishing}{%
\subsection{Open access publishing}\label{open-access-publishing}}

All types of scientific publications created in the context of
NFDI4Microbiota are meant to be published in Open Access journals. In
addition, we urge our users to use preprint servers like
\href{https://www.biorxiv.org/}{bioRxiv} to make their findings openly
accessible prior to publication. Researchers using the services of
NFDI4Microbiota are highly encouraged to publish in Open Access journals
as well.

\begin{quote}
Research findings funded by public money should be made freely available
in Open Access journals to maximize the potential audience and promote
Open Science goals. Especially DFG funded projects have the obligation
to publish their result under open licenses in Open Access and must
contain a reference to the respective project number. In the case of
publications enabled by NFDI4Microbiota this is ``Deutsche
Forschungsgemeinschaft (DFG) - Project number 460129525''.
\end{quote}

\hypertarget{open-data}{%
\subsection{Open data}\label{open-data}}

Data handled/produced in the context of NFDI4Microbiota must be made
FAIR and openly available at the earliest possible time under a
permissive license unless legal etc. restrictions prohibit a publication
in commonly used repositories. This includes the raw data of the
experiments and also the accompanying metadata in a commonly used
metadata standard.

\begin{quote}
Services that receive data should encourage the user to publish their
data openly right away (e.g.~by default) either within the service or
recommended repositories. As a guide to sharing microbiome related data
and analysis results please refer to
\href{https://doi.org/10.1038/s41564-023-01484-x}{Huttenhower et
al.~2023}. Where data is not made publicly available, data sharing with
other users should be enabled with fine grained control over read/write
and re-sharing permissions. Finally, all data and its accompanying
metadata (following the
\href{https://github.com/NFDI4Microbiota/MetadataStandards}{minimal
standards}) needs to be deposited in the respective repositories latest
upon publication of the research. An appropriate and compatible license
(\href{https://creativecommons.org/choose/\#}{Creative Commons}
\href{https://creativecommons.org/publicdomain/zero/1.0/}{CC0} or
\href{https://creativecommons.org/licenses/by/4.0/}{CC-BY} is
recommended) needs to be supported by the service and chosen by the
user. The data set should be properly referenced in the publications
using the respective identifiers.
\end{quote}

\hypertarget{support}{%
\section{Support}\label{support}}

\hypertarget{support-contact}{%
\subsection{Support contact}\label{support-contact}}

Each service requires a contact (e-mail address) deposited with
NFDI4Microbiota for support questions via the
\href{https://nfdi4microbiota.de/contact-form/}{help desk} (e.g.~e-mail
support, online issue tracking).

\begin{quote}
Primarily, developers of a service are responsible for the maintenance,
availability, documentation and supporting users with problems and
handling feature requests. Since we aim to interlock the different
services offered within NFDI4Mictobiota (e.g.~in common workflows), we
want to also offer central support to users who use NFDI4Microbiota as a
platform. In order to address requests directly from the help desk or
delegate issues to the respective maintainers, a contact address and
link to the documentation is needed for each service.
\end{quote}

\hypertarget{documentation-in-the-knowledge-base}{%
\subsection{Documentation in the Knowledge
Base}\label{documentation-in-the-knowledge-base}}

Services maintain a short description with documentation on a sub-page
in the NFDI4Microbiota Knowledge Base published under a permissive
license.

\begin{quote}
As part of the effort to make a service available via the
NFDI4Microbiota platform for better visibility, a short description
(optimally with a screen shot) along with links to the service itself,
documentation and code repository. This sub-page will also be used to
summarize frequently asked support questions and answers by the
maintainers and the help desk.
\end{quote}

\hypertarget{interoperability}{%
\section{Interoperability}\label{interoperability}}

\hypertarget{compatible-community-standards}{%
\subsection{Compatible community
standards}\label{compatible-community-standards}}

Software and services support widely used bioinformatics standard
formats for input and output in order to make them compatible with other
tools. Metadata formats adhere to widely used minimal standards and are
required for any user uploaded data. Services and tools provided by
NFDI4Microbiota which deal with metadata and relevant file formats
support at least one of these standards as a default or recommendation
for users.

\begin{quote}
While a myriad of file formats is generally available and it is easy to
generate purpose-build output for your own applications, many are hard
to comprehend and parse, not or barely specified and generally a
hindrance to re-usability and interoperability. Additionally, neither
metadata, nor provenance information is stored along the primary data
more often than not. Therefore, services are supposed to adhere to
widely-used and specified file formats for input and output. Summary
statistics and logs with complete parameters are highly encouraged to
enable provenance tracking. Web services which let users upload data
should request metadata in a format compatible with at least one
well-specified metadata standard by default.
\end{quote}

\hypertarget{persistent-identifiers}{%
\subsection{Persistent Identifiers}\label{persistent-identifiers}}

Whenever possible, persistent identifiers are used within the services
own databases and when referring/linking to external sources also in
output of offline tools. The interoperability between offered services
that are relevant for each other is highly encouraged whenever possible.

\begin{quote}
Databases have to make use of existing (persistent) identifiers when
importing external resources (e.g.~other databases). External
identifiers should link back to the respective resource via permalinks
or API. Additionally, databases may provide their own persistent
identifiers, especially when cataloging novel entries. These identifiers
should be exposed as permalinks or in an API.
\end{quote}

\hypertarget{re-usability}{%
\section{(Re-)Usability}\label{re-usability}}

\hypertarget{terms-of-use}{%
\subsection{Terms of use}\label{terms-of-use}}

Services publicly state the terms of their use.

\begin{quote}
Web services present their terms of use for visitors as part of their
website to inform users about the conditions and restrictions of the
service, specifying the following information: Who is offering the
service under which license or jurisdiction and with which (limited)
guarantees to which audience (in case of restricted access).
Additionally, services that handle user data (login information or
uploaded material) need to display a privacy policy (ger.:
Datenschutzerklärung) at registration/upload specifying which data is
stored where, the responsible data privacy office and terms for data
retention, sharing and deletion in compliance with the law. Services may
make use of generic Terms of Service and Privacy Policy offered by
NFDI4Microbiota internally and customize them to their needs under legal
advisement.
\end{quote}

\hypertarget{accessibility}{%
\subsection{Accessibility}\label{accessibility}}

Services strive to be accessible by providing APIs (application
programming interface) or supporting machine-readable formats and
support barrier-free accessibility.

\begin{quote}
In order to maximize the usability of a given service, developers must
think of different ways, user may interact with their interface,
sometimes differently than intended. Instead of using a service just as
a front-end to a database, developers should consider exposing an API or
exporting machine-readable data dumps or releases that other developers
can interact or make use of. Additionally, user interfaces should be
designed with specific user constrains in mind, by offering alt-text for
images (to be compatible with screen readers) and avoiding color-only
encoding of information to accommodate red-green deficiency or
color-blindness. Making a user interface rather static than dynamic
(avoid hover-over menus) and navigable by using a logical tab order
improves the usability of all potential users.
\end{quote}

\hypertarget{appendix}{%
\section{Appendix}\label{appendix}}

Links to other consortia's policies and guidelines:

\begin{itemize}
\tightlist
\item
  \textbf{dataPLANT:}
  https://nfdi4plants.org/content/news/2021-11-08-dataplant-tools-and-services-development-principles.html
\item
  \textbf{nfdi4health:}
  https://repository.publisso.de/resource/frl:6431645/data
\item
  \textbf{NFDI4BIOIMAGE:} https://nfdi4bioimage.de/en/aims/
\item
  \textbf{NMDC:} https://microbiomedata.org/data-management/
\item
  \textbf{GHGA:} https://zenodo.org/record/6828131
\end{itemize}

\end{document}
